\section{Data representation}
Inside databases and datasets we can identify:
\begin{itemize}
\item \textbf{Instances} $\to$ observations/cases/records that represent atomic elements of information
\item \textbf{Attributes} $\to$ variables/features that measure aspects of an instance. Each instance has a certain number of attributes
\item \textbf{Concepts} $\to$ things that can be learned inside the data
\end{itemize}

\subsection{Attribute types}
Data types are not only important for our understanding but some algorithms work better with a certain type of data : make it easier to make adequate comparisons for example. \\
Knowing the data type is also important the check for \textbf{valid values} and deal with \textbf{missing values}.
\begin{description}
\item[Numeric attributes]\hfill\\
\begin{itemize}
\item Real-valued or integer-valued domain
\item Interval-scaled when only differences are useful $\to$ temperature
\item Ratio-scaled when only rations are meaningful $\to$ age
\end{itemize}
Numerical attributes are \textbf{ordered} and measured in fixed units.\\
Zero point is only defied for \textbf{ratio attributes}.\\
Can be divided into \textit{discrete or continuous}.
\item[Categorical attributes]\hfill\\
\begin{itemize}
\item Set-valued domain composed of a set of symbols
\item \textit{Nominal}\\ When only equality is meaningful
Values are distinct symbols that serve as labels. No relation is implied among nominal values and only equality test can be performed.
\item \textit{Ordinal}\\When both equality and inequality are meaningful. As in the nominal case, also here talking about \textbf{difference} doesn't make sense.
\end{itemize}
\item[Binary attributes]\hfill\\
Represented by either 0/1

\end{description}
Sometimes the same attribute can be either considered \textbf{nominal} or \textbf{ordinal} : \texttt{if age == young AND ...} is nominal whereas \texttt{if age < presbyopic AND...} is ordinal.\\

\subsection{Missing values}
Many databases present \textbf{missing values} . The nature of missing values is very broad and must be understood : it can be due to faulty equipment, incorrect measurements, censored or anonymous data  ecc...\\
Missing values can have their own importance ( ex: a missing test ) but in that case is should have its own coding.
Usually missing values are indicated by \textbf{out of range } entries, \textbf{Nan} or \textbf{special values}. 
